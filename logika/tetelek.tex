\documentclass[10pt,a4paper]{article}
\usepackage[utf8]{inputenc}
\usepackage[T1]{fontenc}
\usepackage{amsmath}
\usepackage{amsfonts}
\usepackage{amssymb}
\usepackage{mathrsfs}
\author{Aradi Patrik}
\title{Logika Tételkidolgozás}
\begin{document}
\maketitle
\newpage
\section{Az ítéletkalkulus szintaxisa és szemantikája. Kielégíthetőség, logikai következmény, alap összefüggések.}

Az ítéletlogikában a változók a {0, 1} halmazból kapnak értéket. \newline A formulák változókból épülnek fel, melyeket össekötő jelek alkalmazásával kapunk. Pl.: $\neg, \wedge, \vee, \rightarrow, \leftrightarrow$
\paragraph{Szintaxis}
A logikában meghatározza, hogy hogy néz ki egy formula. \newline Pl.: $(p \vee q)$
\paragraph{Szemantika}
Megmondja, hogy a leírt formulának mi a jelentése. Mi az adott formulának az értéke egy adott változó értékadás esetén
\subsection{Szintaxis}
\paragraph{Változók}
p, q, r -el jelöjük, melyek {0, 1} értéket vehetnek fel.
\paragraph{Logikai konstansjelek}
(0 aritású függvényjelek) az "igaz" $\uparrow$ és a "hamis" $\downarrow$ jelek
\paragraph{Konnektívák}
Velük tudjuk összekötni a formulákat, lehetséges értékeik: $\wedge, \vee, \neg, \rightarrow, \leftrightarrow $
\paragraph{Formulák Deffiníciója}
\begin{itemize}
\item Minden változó és minden logikai konstans formula
\item Ha $F$ formula, akkor $(\neg F)$ is formula
\item Ha $F$ és $G$ formulák, akkor $(F \wedge G), (F \vee G), (F \rightarrow G), (F \leftrightarrow G)$ is formulák
\item Más forumla nincs
\end{itemize}
\paragraph{Műveleti sorrend}
A $\wedge$ és $\vee$ műveletek \textit{asszociatívak}, pl.: $(F \vee G) \vee H$ helyett $F \vee G \vee H$-t írhatunk. A $\rightarrow$ művelet \textit{jobb-asszociatív}, $F \rightarrow G \rightarrow H = F \rightarrow ( G \rightarrow H)$ zárójelezést jelenti
\subsection{Szemantika}
\paragraph{Boole-függvény}
Hogy a konnektívák szemantikájáról tudjunk beszélni, mindhez rendelünk egy Boole-függvényt. A Bool-függvény bitvektort egy bitbe képző függvény: $f: \{0,1\}^{n} \rightarrow \{0,1\}.$ \newline
Az $f|n$ jelzi, hogy $f$ egy n-változós függvény. A $\neg$ unáris Boole függvény. A bináris konnektívákhoz rendelt Boole-függvényekhez készíthető igazságtábla. Egy n változós  Boole-függvény $2^{n}$ soros
\paragraph{Értékadás}
Egy $\mathcal{A}$ függvény, mely minden változóhoz egy igazságértéket (bitet: 0 vagy 1) rendel. Egy formula kiértékeléshez szükség van egy értékadásra. \newline
\paragraph{Az $\mathcal{A}$ értékadás mellet az $F$ formula értékét $\mathcal{A}(F)$ jelöli.}
\begin{itemize}
\item Ha a formula a $p$ változó, akkor értéke $\mathcal{A}(p)$
\item $\mathcal{A}(\uparrow) = 1$
\item $\mathcal{A}(\downarrow) = 0$
\item $\mathcal{A}(\neg F) = \neg \mathcal{A}(F)$
\item $\mathcal{A}(F \vee G) = \mathcal{A}(F) \vee \mathcal{A}(G)$
\item $\mathcal{A}(F \wedge G) = \mathcal{A}(F) \wedge \mathcal{A}(G)$
\item $\mathcal{A}(F \rightarrow G) = \mathcal{A}(F) \rightarrow \mathcal{A}(G)$
\item $\mathcal{A}(F \leftrightarrow G) = \mathcal{A}(F) \leftrightarrow \mathcal{A}(G)$
\end{itemize}
Tehát rekurzívan kiértékeljük az "eggyel egyszerűbb" formulákat és a legkülső konnektívának megfelelően kombináljuk az értékeket.
\paragraph{Közvetlen részformula}
Egy formula közvetlen részformulái az "eggyel lentebbi szinten lévő részei". 
\begin{itemize}
\item Változóknak és a logikai konstansoknak nincs közvetlen részforulája.
\item A $(\neg F)$ alakú formulák közvetlen részformulája $F$
\item Az $(F \vee G), (F \wedge G), (F \rightarrow G), (F \leftrightarrow G)$ alakú formulák közvetlen részformulái  $F$ és $G$
\end{itemize}
A formulák kiértékelését úgy végeztük el, hogy rekurzívan kiértékeljük a közvetlen részformulákat, majd az eredményekből és a külső konnektivitásból számítjuk az egész formula értékét. Az ilyen rendszerű definíciókat és bizonyításokat a \textbf{formula felépítése szerinti teljes indukció}nak nevezzük.
\paragraph{Felépítés szerinti indukció}
Deffiníciókban csak meg kell mondjuk, hogy aktuálisan a formulához rendelt objektumot hogyan számítjuk ki a részformuláihoz rendelt objektumokból, ügyelve arra, hogy minden esetet pontosan egyszer vegyünk sorra. \newline
Bizonyításokban minden esetre meg kell mutatnunk, hogy ha az állítás igaz a formula összes közvetlen részformulájára, akkor miért igaz az egész formuára is. (teljes indukció is így működik)
\paragraph{Kielégíthetőség}
Ha az $\mathcal{A}$ értékadásra és az $F$ formuálra $\mathcal{A}(F)=1$, azt úgy is írjuk, hogy $\mathcal{A}\models F$ és úgy is mondjuk, hogy $\mathcal{A}$ kielégíti $F$-et, vagy $\mathcal{A}$ egy modellje $F$-nek. Ha egy formulának van modellje, akkor azt mondjuk, kielégíthető, ha nincs, kielégíthetetlen. Ha az $F$ formulának minden kiértékelés modellje, akkor tautológia, ennek jele pedig $\models F$.
\paragraph{Modellek halmaza}
Ha $F$ egy formula, akkor $Mod(F)$ az $F$ összes modelljének halmaza. Tehát azt hogy $\mathcal{A}(F)=1$, vagy $\mathcal{A} \models F$, úgy is írhatjuk, hogy $\mathcal{A}\in Mod(F)$. $F$ pontosan akkor kielégíthetetlen, ha $Mod(F)=\emptyset$. Ha $\Sigma$ formulák egy halmaza és $\mathcal{A}$ egy értékadás, akkor $\mathcal{A} \models \Sigma$ azt jelenti, hogy $\mathcal{A}$ kielégíti $\Sigma$ összes elemét. $F$ formula pontosan akkor tautológia, ha $\neg F$ kielégíthetetlen.
\paragraph{Logikai következmény}
Ha $F$ és $G$ formulák, akkor $F \models G$ ("$F$-nek következménye $G$") azt jelöli, hogy minden $\mathcal{A}$-ra ha $\mathcal{A}(F)=1$, akkor $\mathcal{A}(G)=1$. Tehát, ha $F$ igaz akkor G is igaz, és $Mod(F) \subseteq Mod(G)$
Ugyanígy használhatjuk a $\Sigma \models F, \Sigma \models \Gamma$ jelöléseket is, ahol $\Sigma, \Gamma$ formulahalmazok. Pl.: $\Sigma \models F$ akkor áll fenn, ha $\Sigma$ minden modellje, modellje $F$-nek is. $F \equiv G$ jelölés azt jelenti, hogy $Mod(F) = Mod(G)$
\paragraph{Tétel $Mod(\Sigma \cup \Gamma) = Mod(\Sigma) \cap Mod(\Gamma)$}  Hiszen a bal oldalon szereplő halmazban azok az értékadások vannak, melyek kielégítik $\Sigma \cup \Gamma$ összes elemét, azaz $\Sigma$ összes elemét is és $\Gamma$ összes elemét is, azaz melyek benne vannak $Mod(\Sigma)$-ban is és  $Mod(\Gamma)$-ban is, ez pedig épp a jobb oldal.
\paragraph{Fenti bizonyítható indirekt módon is}


\newpage
\section{Boole-függvények. Shannon-expanzió. Boole-függvények teljes rendszerei}
\paragraph{Boole-függvény}
A Bool-függvény bitvektort egy bitbe képző függvény: $f: \{0,1\}^{n} \rightarrow \{0,1\}.$ Ha az $F$ formulában csak a $\{p_{1}, \ldots, p_{n}\}$ változók szerepelnek, akkor $F$ indukál egy n változós Boole-függvényt, melyet szintén $F$-fel jelölünk. Például, ha a formula $p_{i}(x_{1}, \ldots, x_{n})$ akkor egy olyan Boole-függvényt fog indukálni, hogy $n$ darab bit bejön, és kiválasztja az $i.$ bitet.
\begin{itemize}
\item $p_{i}(x_{1}, \ldots, x_{n}) = x_{i}$ (ezt projekciónak hívjuk)
\item $(\neg F)(x_{1}, \ldots, x_{n}) = \neg(F(x_{1}, \ldots, x_{n}))$
\item $(F \vee G)(x_{1}, \ldots, x_{n}) = F(x_{1}, \ldots, x_{n}) \vee G(x_{1}, \ldots, x_{n})$
\item $\ldots$
\end{itemize}
\paragraph{Boole-függvények megszorításai} Legyen $f|n$ Boole-függvény, $n>0$. Ha $b \in \{0,1\}$ igazságérték, úgy hogy $f\vert_{x_{n} = b}$ jelöli azt az $(n-1)$-változós Boole-függvényt, melyet úgy kapunk, hogy $f$ inputjában $x_{n}$ értékét $b$-re rögzítjük. Formálisan:
\begin{equation*}\label{eq:pareto mle2}
  \begin{aligned}
f\vert_{x_{n} = b}(x_{1}, \ldots, x_{n-1}) = f(x_{1}, \ldots, x_{n-1}, b)
 \end{aligned}
\end{equation*}
Példák:
\begin{itemize}
\item $\vee \vert _{x_{2}=1}$ a konstans 1 függvény: $\vee \vert _{x_{2}=1}(x_{1}) = \vee(x_{1}, 1) = 1$.
\item $\wedge \vert _{x_{2}=0}$ a konstans 0 függvény
\item $\wedge \vert _{x_{2}=1}$ az identikus $(x_{1} \rightarrow x_{1})$ függvény
\end{itemize}
\paragraph{Shannon expanzió}
Lényege, hogy egy n változós Boole-függvényt ki tudunk fejezni két n-1 változós Boole-függvény segítségével. Bejön az $f(x_{1}, \ldots, x_{n})$, két eset lehetséges: az $x_n$ vagy 1 vagy 0. Az alábbi képlet esetén:
\begin{equation*}\label{eq:pareto mle2}
  \begin{aligned}
f(x_{1}, \ldots, x_{n}) = (x_n \wedge f\vert_{x_n = 1}(x_{1}, \ldots, x_{n-1})) \vee (\neg x_n \wedge f\vert_{x_n = 0}(x_{1}, \ldots, x_{n-1}))
 \end{aligned}
\end{equation*}
Ha az $x_{n}=1$, akkor a jobb oldali tag hamis lesz, a bal oldal pedig: $f\vert_{x_{n} = 1}(x_{1}, \ldots, x_{n-1}) = f(x_{1}, \ldots, x_{n-1}, 1)$, ami megegyezik $f(x_{1}, \ldots, x_{n})$-nel ez esetben.
Minden Boole-függvény előáll a projekciók és a $\{\neg, \vee, \wedge\}$ Boole-függvények alkalmas kompozíciójaként. Ezt úgy is mondjuk, hogy $\{\neg, \vee, \wedge\}$ rendszer teljes.
\paragraph{Bizonyítás} $n$ szerinti teljes indukciót alkalmazunk. Ha $n = 0$, akkor $f|0$ függvény vagy konstans 0, vagy a konstans 1, mindkettő előállítható így. Ha $n > 0$, akkor az indukciós feltevés zserint az $f\vert_{x_{n} = b}(x_{1}, \ldots, x_{n-1})$ Boole-függvények $b \in \{0,1\}$-re előállnak ilyen alakban, a Shannon expanzióban pedig szintén csak ezt a három műveletet alkalmazzuk.
\paragraph{Következmény} Minden Boole-függvény indukálható olyan formulával, melyben csak $\{\neg, \vee, \wedge\}$ konnektívák szerepelnek.

\newpage
\section{Konjunktív normálforma. A DPLL algoritmus}
\paragraph{Konjunktív normálforma}
Az egyik leggyakrabban alkalmazott normálforma a konjuktív normálform. Az ítéletváltozókat és negáltjaikat literáloknak nevezzük. Véges sok literál diszjunkcióját klóznak nevezzük. Véges sok klóz konjunkcióját pedig konjunktív normálformának, CNF-nek.
\end{document}
